\documentclass[12pt]{article}
\input{vc}

\begin{document}

\section*{Likelihood-free inference \\ for large-scale structure projects}
\noindent
\texttt{draft version {\githash~(\gitdate)}}

\bigskip
\noindent
by DWH and others

\bigskip
\begin{abstract}
The traditonal methods for cosmological parameter inference in
large-scale structure surveys involve estimation of a two-point
function (correlation function or power spectrum) and then
construction of a chi-squared loss function or approximate Gaussian
likelihood function based on that estimator.
Even in the approximation or limit in which the two-point function is
a sufficient statistic for all cosmological parameter estimation, the
contruction of the approximate likelihood function is not justifiable;
that is, there is no set of assumptions under which cosmological
inferences based on this approximate likelihood function are expected
to be correct in detail.
Here we build a likelihood-free inference (LFI), or approximate
Bayesian computation (ABC), for large-scale structure surveys that
makes fewer assumptions than the traditional analyses, and which
returns correct posterior inferences about the cosmological parameters
under a well defined set of assumptions.
That is, the LFI is protected by proofs of correctness, unlike the
traditional methods.
The method relies on the existence of simulation code that can
generate realistic mock survey data, and involves adaptive sampling
from priors over cosmological parameters.
With good choices, the likelihood-free method is comparably expensive
to the contemporary implementations of the traditional methodology in
terms of computational resources.
We perform tests with mock data and SDSS-III BOSS data to demonstrate
and validate the LFI method.
\end{abstract}

\section{Introduciton}

...Cartoon cosmology:  Estimate $\xi$, estimate covariance matrix, construct artificial likelihood function (or really $\chi^2$ function).

...It is not possible for there to be a Gaussian LF for the correlation function.

...LFI / ABC:  Try to find models that ``exactly'' reproduce certain statistics of the data.

...Conditions under which LFI is actually correct.

\section{Data}

\section{Method}

\section{Experiments and results}

...One experiment on fake data showing that LFI is less biased (is it?)\ than the Cartoon Cosmology method.

...One experiment on real data showing that we get slightly different answers.

\section{Discussion}

...LFI works and is better than the CC method, both theoretically (in terms of proofs) and experimentally (it shows smaller bias in tests on fake data).

...Compare computational costs.

...Mention code release.

...Advocate a switch.

\end{document}
